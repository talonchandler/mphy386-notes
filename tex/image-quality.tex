\documentclass[mphy386-notes.tex]{subfiles}
\begin{document}
\section{X-ray Image Quality}
In this section we will introduce the basics of x-ray imaging and develop three
tools that we can use to quantify image quality---contrast, resolution, and
noise. We will focus on x-ray imaging in these notes, but these tools are
useful for analyzing all imaging modalities.

\subsection{Review of useful concepts}
\subsubsection{Central-limit theorem}
\subsubsection{Gaussian random variable}
\subsubsection{Poisson statistics}
\subsubsection{Stationarity}
\subsubsection{Ergodicity}
\subsubsection{Ensemble average}

\subsection{Resolution}
\subsubsection{Signal transfer / linear system model}

\subsection{Noise}
\subsubsection{Noise transfer (autocorrelation, NPS)}

\subsection{Contrast}
\subsubsection{Rose Model}

\subsection{The effect of Scatter}

INTRODUCTION, so far we have neglected scatter, etc.

\subsubsection{Resolution: Scatter psf}

Linear systems analysis

\begin{align}
  \text{PSF}_{total} (\vec{r}) = \text{PSF}_{primary}(\vec{r}) + \text{PSF}_{scatter}(\vec{r})
\end{align}

Modeling of Scatter PSF

\begin{align}
  \text{PSF}_{scatter}(\vec{r}) = \int_0^L dz\left[\Phi_0 e^{-\mu(L-z)}\right]\left[\frac{r_0^2}{2}\left(1 + \text{cos}^2\theta\right)\right]\left[n_e A\right]\left[\frac{\text{cos}^3\theta}{(s+z)^2}\right]\left[e^{-\mu z\text{sec}\theta}\right]
\end{align}

\begin{align}
  \text{PSF}_{primary}(\vec{r}) = \Phi_0 e^{-\mu L}
\end{align}

Modeling Scatter as Gaussian

\begin{align}
  \text{PSF}_{scatter}(\vec{r}) = A_s T e^{-\pi \beta_s^2 r^2} & & \text{P}_{scatter}(\vec{\rho}) = \frac{A_s T}{\beta_s^2}e^{-\pi \rho^2 / \beta_s^2}\\
  \text{PSF}_{primary}(\vec{r}) = A_p T e^{-\pi \beta_p^2 r^2} & & \text{P}_{primary}(\vec{\rho}) = \frac{A_p T}{\beta_p^2}e^{-\pi \rho^2 / \beta_p^2}
\end{align}

\begin{align}
  \text{MTF}_{total} &= \frac{\text{P}_{primary}(\vec{\rho}) + \text{P}_{scatter}(\vec{\rho})}{\text{PSF}_{primary}(0) + \text{PSF}_{scatter}(0)}\\
  &= \frac{e^{-\pi\rho^2/\beta_p^2} + \text{SPR}e^{-\pi\rho^2/\beta_s^2}}{1 + \text{SPR}}
\end{align}

Important results
\begin{align}
  \frac{\text{P}_{total}(\vec{\rho})}{\text{P}_{primary}(\vec{\rho})} \approx \frac{1}{1 + \text{SPR}} & & \beta_s << \vec{\rho} << \beta_p
\end{align}

\subsubsection{Contrast: Rose Model}

\fig{img/scatter.png}{1.0}{Left: Radiograph of the skull with contrast. Right:
  Radiograph of the same skull with an anti-scatter grid in place. Note the
  improvement in image contrast.}{scatter}
% I'm assuming an anti-scatter grid is the difference between left and right
% images, I could be wrong...

The derivation of expressions for contrast and SNR under the Rose model in
section 2.2 neglected the effects of scattered radiation. In general,
scattered radiation always works to decrease the contrast and overall
quality of an image, as seen in Figure \ref{fig:scatter}.
This effect can be quantitatively demonstrated with a simple extension
of the Rose model. 

% Will make a different figure in the future similar to the one in section 2.2
\fig{img/rose_with_scatter.png}{0.3}{Rose model with scatter.}{rose_with_scatter}

Recall our previous definition of contrast under the Rose model,
\begin{align}
  C = \frac{\Delta \phi}{\bar{\phi}} = \frac{\phi_1 - \phi_2}{(\phi_1 + \phi_2)/2}
\end{align}
Referring to Figure \ref{fig:rose_with_scatter}, we can define a
``no scatter'' contrast, $C_{NS}$, accounting only for contrast from
the primary fluence components, $\phi_{1p}$ and $\phi_{2p}$, That is,
\begin{align}
  C_{NS} = \frac{\phi_{1p} - \phi_{2p}}{(\phi_{1p} + \phi_{2p})/2} = \frac{\phi_{1p} - \phi_{2p}}{P}
\end{align}
where we define $P$ as the mean primary fluence, $(\phi_{1p} + \phi_{2p})/2$.

Likewise, we can define a ``scatter'' contrast, $C_S$,
\begin{align}
  C_S &= \frac{(\phi_{1p} + \phi_{1s}) - (\phi_{2p} + \phi_{2s})}{[(\phi_{1p} + \phi_{1s}) + (\phi_{2p} + \phi_{2s})]/2}\\
      &= \frac{(\phi_{1p} - \phi_{2p}) + (\phi_{1s} - \phi_{2s})}{[(\phi_{1p} + \phi_{2p}) + (\phi_{1s} + \phi_{2s})]/2}\\
\end{align}

If we assume that the scatter component of each fluence is roughly equal, that is $\phi_{1s} \approx \phi_{2s} = \phi_S$, and define $S = \phi_S$, then $C_S$ can be written
\begin{align}
  C_S = \frac{\phi_{1p} - \phi_{2p}}{P + S}
\end{align}
and we see that scatter reduces contrast.

We can relate the scatter and no-scatter contrasts as follows:
\begin{align}
  C_S &= \frac{\phi_{1p} - \phi_{2p}}{P + S}\left(\frac{P}{P}\right)\\
      &= C_{NS}\left(\frac{P}{P + S}\right)\\
      &= C_{NS}\left(1 - \frac{S}{P + S}\right) \\
      &= C_{NS}(1 - \text{SF}) = C_{NS}\left(\frac{1}{1 + S/P}\right)
\end{align}
using the scatter fraction:
\begin{align}
  \text{SF} = \frac{S}{P + S}
\end{align}
and scatter-to-primary ratio, $S/P$. 

\subsubsection{Anti-scatter grids}

\fig{img/antiscattergrid}{0.6}{Anti-scatter grid basics.}{fig:antiscatter}

\pagebreak
\end{document}
