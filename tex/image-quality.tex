\documentclass[mphy386-notes.tex]{subfiles}
\begin{document}
\section{Image Quality}
In this section we develop three important tools that we can use to quantify
image quality---contrast, resolution, and noise. In this section we will discuss
these quantities without referencing a specific imaging modality.


\subsection{Contrast}



\subsection{Resolution}
\subsection{Noise}
\subsection{Rose Model}

The signal is the difference in intensity between the object and its background:
\begin{align*}
  \Delta S = S_1 - S_2 = A\phi_1 - A\phi_2 = A\Delta\phi.
\end{align*}

where $A$ is the cross sectional area of the object, $\phi$ is the x-ray fluence
with units of photons per unit area and $\Delta\phi = \phi_1-\phi_2$.

The noise is the square-root of the variance in background over areas the size
of the object. Assuming the noise follows Poisson statistics, where the
variance means the mean value:
\begin{align*}
  \text{Noise} = \sqrt{\sigma^2} = \sqrt{\frac{A\phi_1 + A\phi_2}{2}} = \sqrt{A\bar{\phi}}
\end{align*}
Therefore, the signal-to-noise ratio is,
\begin{align*}
  \text{SNR} &= \frac{A\Delta\phi}{\sqrt{A\bar{phi}}}\\
  \text{SNR} &= C\sqrt{A\bar{phi}} 
\end{align*}
where C is the radiation contrast:
\begin{align*}
  C = \frac{\Delta\phi}{\bar{phi}}
\end{align*}
This is the SNR for an ideal detector where we've assumed that
\begin{itemize}
\item complete absorption of incident quanta
\item no added noise
\item no loss of spatial resolution (i.e., no blurring)
\end{itemize}

 \fig{img/rose.png}{.25}{Test}{rose}
 \fig{img/contrast.png}{.25}{Test}{constrast}

\pagebreak
\end{document}