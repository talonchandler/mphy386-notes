\documentclass[mphy386-notes.tex]{subfiles}
\begin{document}
\section{Introduction}
There are many different imaging modalities in the radiology department---
mammography, computed radiography, digital radiography, fluoroscopy, computed
tomography (CT), magnetic resonance imaging (MRI), positron-emission tomography
(PET), single-photon emission tomography (SPECT), ultrasound, and more. Why are
there so many modalities? Because no single modality is optimal for the
wide variety of imaging tasks that arise when we try to diagnose disease. Each
imaging task has different imaging requirements, so no single modality can be
used for all tasks.

A major goal of an imaging medical physicist is to choose and optimize the
imaging modality that maximizes the diagnostic content of the image while
minimizing cost. This is not a simple task---diagnostic content depends on each
specific task, and cost depends on the equipment price, the number of patients
that can be imaged per unit time, the radiologists’ interpretation time, the
cost of the exam to the patient, the risks to the patient, the consequences of a
false negative exam, the cost of a false positive exam, and more.

One way to understand a medical physicist's task is to examine the relationship
between cost, image quality, and diagnostic accuracy. If we assume that there is
a monotonic relationship between image quality and diagnostic accuracy (Figure
XXX), and if we assume that there is a similar relationship between image
quality and cost (Figure XXX), then 

In this part of the course, you will learn how to quantify image quality and how
it is affected by various acquisition parameters. First, we will review the mathematics required and
discuss common image quality metrics. Second, we will discuss x-ray imaging
physics and learn how to optimize x-ray imaging systems using the previously
developed metrics. Finally, we will discuss the interpretation of the x-ray images
and how our interpretation fits into the broader context. 

These notes draw heavily from...

Our main reference will be \cite{barrett}. 
\pagebreak
\end{document}