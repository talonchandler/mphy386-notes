\documentclass[mphy386-notes.tex]{subfiles}
\begin{document}
\section{X-ray Imaging}
\subsection{X-ray Imaging Physics}
\subsubsection{Sources}
X-ray tubes.
kVp, mA, etc.
Focal spot size

\subsubsection{Interactions}
Review of interactions in x-ray energy regime
Scatter, grids

\subsubsection{Detectors}
An ideal x-ray detector converts an x-ray image into a visible image without
altering it. In this section we will review the major types of x-ray detectors
and examine how they fail to be ideal.

The first task of any x-ray detector is to absorb all incident x-rays---a
detector should not let any x-rays pass through. The fraction of incident x-rays
that interact in the detector, $\eta$, is called the quantum detection
efficiency. The quantum detection efficiency is related to the linear
attenuation coefficient of the x-ray detector, $\mu$, and the thickness of the detector $t$ by
\begin{align}
  \eta = 1 - e^{-\mu t}. 
\end{align}
An ideal detector will stop all x-rays, so we would like $\mu t$ to be as large
as possible. Therefore, an x-ray detector should be thick and made of a material
with high atomic number and high physical density. If the quantum detection
efficiency is less than one, then the detector's SNR will be reduced by
$\sqrt{\eta}$ and Eq. XXX becomes
\begin{align}
  \text{SNR} = C\sqrt{\eta A\bar{\phi}}.
\end{align}
Recall from the Rose model that we need to an SNR of approximately 5 to detect a
signal on a background. This means that reducing the quantum efficiency will
require us to increase the incident x-ray fluence to maintain a fixed SNR.
Increasing the photon fluence will increase the dose to the patient, so clearly
quantum efficiency is an important consideration for all detectors.

\subsubsection*{Analog Detectors}
Analog images are acquired and displayed on film. Film consists of a plastic
base coated with a gelatin binder that contains light-sensitive silver halide
crystals. Figures \ref{fig:film} and \ref{fig:sem} show schematic and scanning
electron micrograph views of film.

When a photon hits a (transparent) silver halide crystal, a photochemical
reaction occurs that creates two (opaque) metallic silver atoms. Under typical
conditions the number of metallic silver atoms is far too small for the film to
become visibly opaque to human eyes, but the film now contains a latent image in
the form of silver atoms within silver halide grains. To make the image visible,
we need to ``amplify'' the number of metallic silver atoms in each grain using a
\textit{developer solution}, then remove the unexposed silver halide crystals
using a \textit{fixing solution}. The result is a \textit{photographic negative}
that is dark (lots of metallic silver) in regions that have been exposed and
light in regions that have not been exposed. In x-ray imaging the photographic
negative is used directly, but in photographic imaging you need to invert the
film brightness by projecting light through the negative, imaging the result on
film, then repeating the development process. See \cite{barrett, stack} for a more
detailed discussion of the film development process.

For an x-ray film with a thin coat of silver halide grains, the grain size
determines the spatial resolution. Film grains are approximately 0.2-2~$\mu$m in
diameter which gives us a sense of the smallest resolvable feature. For
comparison a typical 2018 scientific CCD has a pixel width of 6 $\mu$m and an
iPhone X display has a pixel width of 55~$\mu$m. Film has unmatched spatial
resolution even in 2018.

Only a few percent of incident x-rays interact as they pass through film.
Because quantum efficiency is so important for medical imaging, most medical
x-ray film uses a photographic emulsion coating on both sides of the plastic
base. We could increase the thickness of the emulsion layers to improve quantum
efficiency further, but this would reduce spatial resolution and make developing
the film more difficult.

We can improve the quantum efficiency of film by placing a \textit{fluorescence
  screen} directly on the film. Fluorescence screens have a high attenuation
coefficient and contain an x-ray phosphor that converts incident x-rays into
many visible (plus infrared and ultraviolet) photons. These visible photons
expose the film and the film can be developed normally. Unfortunately, the
visible photons do not travel straight to the film along their generating
x-ray's path---they spread out before they reach the film. This means that using
a fluorescence screen improves quantum efficiency at the expense of spatial
resolution. This type of detector is often called an \textit{indirect analog}
detector because it indirectly detects the incident x-rays.

\fig{img/film.jpg}{.7}{Schematic view of photographic film.
  \cite{stack}}{film}

\fig{img/sem.png}{0.7}{Scanning electron micrographs (SEM) of photographic film.
  \textbf{A}: A top-down SEM image of the emulsion layer of T grain emulsion.
  \textbf{B}: A cross section of the T grain emulsion film showing the grains in
  the gelatin layer supported by the polymer film base below. \textbf{C}: Cubic
  grain film emulsion in a top-down SEM view. \textbf{D}: Cross section of the
  cubic grain film. SEM courtesy of Drs. Bernard Apple and John Sabol. Citation
  needed.}{sem}

The input to a film detector is the x-ray exposure measured in coulombs per
kilogram (C/kg) or r\"{o}ntgen (R). The output of a film detector is usually
measured with a unitless property called optical density (OD). If a thin beam of
light with intensity $I_0$ hits the film and the transmitted beam has intensity
$I$, then the optical density of the film at that point is defined as
\begin{align}
  \text{OD} = -\ln\left(\frac{I}{I_0}\right).\label{eq:od}
\end{align}
A high OD means that the film is opaque which corresponds to a high x-ray
exposure. The relationship between x-ray exposure (input) and optical density
(output) is often called the \textit{characteristic} curve, the
\textit{Hurter-Driffield} curve, or the $D$---$\log E$ curve. An example of a
characteristic curve is shown in Fig. \ref{fig:hd}. Note the often confused
units on the characteristic curve---the horizontal axis uses a logarithmic
exposure scale and the vertical axis uses a linear scale for optical density
which is itself a logarithm of the input-output intensity ratio (see Eq.
\ref{eq:od}).

\fig{img/hd-curve.png}{0.5}{A characteristic curve for film. Notice the
  logarithmic scale on the horizontal axis and the linear scale on the vertical
  axis.}{hd}

Let's examine the features of the characteristic curve in Fig. \ref{fig:hd}.
When the film has not been exposed the optical density is not zero---even
unexposed film is not completely transparent due to reflections, impurities, and
thermal effects. As we start to expose the film, we begin to create metallic
silver atoms. A single metallic silver atom in a grain is not enough to create a
latent image though; about 4 metallic silver atoms are required in a single
grain to ensure that it will be developed. This means that for low exposures the
optical density will increase slowly---this is evident in the ``toe'' of the
characteristic curve. At high exposures most of the grains in the film already
have many metallic silver atoms, so increasing the exposure will not increase
the optical density further---this is evident in the ``shoulder'' of the
characteristic curve. Finally, at intermediate exposures the optical density
increases linearly with the log exposure. The slope of characteristic curve is
often called the \textit{speed} of the film because it summarizes how quickly a
fixed exposure rate will increase the optical density---this terminology is
borrowed from photographers who will use a ``fast'' film in situations where
they require short exposure times e.g. sports photography. See \cite{barrett}
chapter X for a complete derivation of film's characteristic curve.

The characteristic curve also depends on the film processing conditions which
can vary significantly. The proportions of chemicals in the developer solution
and the temperature will affect the shape of the characteristic curve and the
speed point. Care is taken to keep the chemistry and the temperature of the
developer as constant as possible.

Film's characteristic curve also depends on the exposure rate. The metallic
silver atoms in each grain are unstable and will re-ionize back into the silver
halide crystal over time. If the exposure rate is very low, then these
re-ionization events can significantly reduce the number of developable grains
which will reduce the OD in the final image. In other words, the characteristic
curve will shift to the right for low exposure rates. This property of film is
often referred to as \textit{reciprocity law failure} or the
\textit{Schwarzschild effect} after Schwarzchild noticed the issue while imaging
dim stars.

\subsubsection*{Digital Detectors}
We will briefly discuss several types of digital detectors and leave the details
for Patrick's lectures. The defining feature of a digital detector is that its
output is a digital signal with discrete values. Digital images are usually
displayed electronically, but they can be also printed on film. Thus, the
acquisition of the image is separate from the display of the image and each
component can be optimized separately. The input to a digital detector is
exposure, and the output is a pixel value (PV). Digital systems usually respond
linearly to x-ray exposure.

\textit{Indirect digital detectors} use a phosphorescent screen (often CsI)
coupled to a charge-couple device (CCD) camera or a think-film transistor (TFT)
water. Incident x-rays are converted to visible photons in the screen which
generates a signal in the CCD. Indirect digital detectors are analogous to
indirect analog detectors.

\textit{Direct digital detectors} use materials that produce electron-hole pairs
that can be collected directly. Selenium (Se), amorphous silicon (a-Si), and
cadmium zinc telluride (CZT) are common materials for direct digital detectors.
An electric field is applied across the width of the detector and the
electron-hole pairs follow the field lines that are perpendicular to the surface
of the detector where they are collected and read out.

\textit{Photon counting detectors} can measure individual x-ray interactions.
Some direct digital detectors can act as photon-counting detectors. Non-photon
counting detectors integrate quanta (photons or electrons) over the total
exposure time of the image acquisition.

\textit{Computed radiography (CR) detectors} store a latent image as electrons
trapped in a phosphor screen (usually BaFCl). The electrons are subsequently
read out by scanning the phosphor with a laser beam. The laser light stimulates
the trapped electrons back into the conduction band where they can return back
to the valence band with the emission of light. The emitted light is collected
to create a digital image. These detectors use the same principle as optically
stimulated luminescent dosimeters (OSLD) in radiation dosimetry.

\subsubsection{*Comparison of Analog and Digital Detectors} 

Contrast

Speed

Latitude

Resolution

Noise

SNR

The contrast in the image incident on the detector is given (ignoring any
scattered radiation) by the radiation contrast, Eq. [4]. We want to
know the radiographic contrast --- the contrast in the final image. To determine
this, we need to know how the x-ray exposure incident on the detector is converted
to a visible image. This relationship between the input exposure and the output
image is given by the characteristic curve.

b. Characteristic Curve

Gives the relationship between the detector output and the exposure to the
detector. For a digital detector the characteristic curve is linear. That is,
\begin{align}
  \text{PV} = GE
\end{align}
where PV is the pixel value in the image, G is the slope of the
characteristic curve and E is the incident exposure. Further:  % isn't exposure usually 'X'?
\begin{align}
  \frac{\Delta E}{\bar{E}} = \frac{\Delta \text{PV}}{\bar{\text{PV}}} = C.
\end{align}

That is, for a digital detector, the radiographic contrast is equal to the
radiation contrast and this is true for all exposure values. It is independent
of the slope of the characteristic curve.

% For screen-film systems, the characteristic curve is called the H\&D curve, named after Hurter and Driffield, and it is a plot OD versus log relative exposure to the screen.

% For screen-film systems, the response is non-linear and there is a toe, a
% shoulder region, and a linear region in between. Base+fog is the minimum OD due
% to the transparency of the film base and any darkening of the film due to
% thermal effects. The net OD is the gross OD minus the base+fog level.

The characteristic curve for a screen-film system depends on the properties of the
screen-film system and the film processing (developer) conditions.

For a screen-film system, the radiographic contrast is given by the difference in
optical density, D. It depends on the radiation contrast and the slope of the
H\&D curve, called gamma (G). Radiographic contrast in a screen-film image is
given by:
\begin{align}
  G &= \frac{\Delta D}{\Delta(\ln E)}\\
  \Delta D = G(\log_{10}e)\Delta(\ln E) &= G(\log_{10}e)\frac{\Delta E}{\bar{E}} = CG\log_{10}e, 
\end{align}
since $\phi = kE$, then $\frac{\Delta E}{\bar{E}} = \frac{\Delta \phi}{\bar{\phi}} = C =$ radiation contrast.

Since the characteristic curve is not linear, the exposure to the detector is
very important. The image can be properly exposed, but also under or over
exposed, where the radiographic contrast will be low (because G is low).  For
digital system, this is not a problem (at least in terms of contrast) as
illustrated below.

Effect of characteristic curve shape.  Top is for screen-film, which have a
non-linear response. Bottom is for a digital system with a linear response.
This figure only illustrates the effect on radiographic contrast and not noise
nor SNR.

c. Speed 

Speed is defined as the reciprocal of the exposure required to reach a net OD of
1.0. The speed point is considered the exposure to give a properly exposed
image. A fast system has high speed and slow system has low speed.  Screen-film
systems have an optimum exposure that must be used in order to produce a useful
image.

Factors Affecting Speed
1.  X-ray absorption by the screen
- phosphor type (atomic number, k-edge energy)
- thickness and packing density
- x-ray energy
- crystal size
2.  Conversion Efficiency of Screen (fraction of x-ray energy converted in optical energy)
- physical properties of phosphor
- optical properties of screen
- concentration of activator atoms
- x-ray energy
3.  Film Sensitivity
- silver content
- sensitizers
- film gain size, structure, etc.
4.  Matching of light emission of the screen to the spectral sensitivity of the film
5.  Film processing

d.	Latitude

For screen-film systems, since the curve is non-linear, the system has limited
latitude. Latitude refers to the range in exposure that will produce density
within the accepted range for diagnostic radiology (usually considered to be
0.25 to 2.0). Latitude does not apply to digital systems. For screen-film
systems, there is a tradeoff in latitude and contrast. Generally speaking,
systems with high contrast (large G) have limited latitude and vice versa.

For the image on the right, System A has higher speed and wider latitude than
System B. System B has higher contrast, but limited latitude.

Wide latitude is important for imaging tasks where there are large difference in
tissue types. For example a chest image requires that image display lung tissue
(mostly air) and ribs (bone). Wide latitude is required to image both of these
simultaneous with good contrast. With a digital detector, since the response to
x-ray exposure is linear, the display of the image can be manipulated so that
bone can be displayed properly and then lung tissue; or image processing can be
used so that both are imaged optimally in a single image.

e. Resolution In a phosphor screen, x rays are converted to optical photons that
must travel through the bulk of the screen to escape. For screens that are
composed of crystals of phosphor in a binder material (turbid screen), the light
is scattered multiple times as illustrated below and light can be absorbed in
the screen. The light at the output of the screen is spread over a finite area,
reducing spatial resolution. The scattering of light in the screen increases
spatial resolution because it preferentially reduces light photons that travel a
long distance. Recall that resolution can be characterized by the point spread
function (psf) and the modulation transfer function (MTF). The image below
gives a qualitative depiction of how the scattering of light broadens the psf
and thus reduces the high frequency components of the MTF.

The spatial resolution is reduced (more spread of light) as the thickness of the
screen increases. The further the distance the light needs to travel to exit
the screen, the broader the psf will be. In many instances a film is sandwiched
between two thinner screens rather than be used with a single thick screen.
This can improve the resolution compared to using a single thick screen. It is
important that the screen and film be in close contact, as any space (poor
contact) will increase the area over which the light has spread.

In CsI phosphor, the crystals of CsI form long needle shaped structures. These
“needles” act like an optical fiber reducing the lateral spread of light
improving the resolution compared to turbid screens of equal thickness.

For direct digital detectors, the spatial resolution can be very high. An
electric field can be placed across the photoconductor forcing the electrons to
travel in direction perpendicular to the surface of the detector greatly
reducing the lateral spread of the electrons.

f. X-ray Quantum Noise

The signal in a screen-film system, the signal is radiographic contrast, as
given in Eq. [9]. The noise in as screen-film image is $\sigma_D$, and it is
related to the noise in the x-ray image incident on the detector, $\sigma_E$.

For a uniform exposure, we can average the square of $\Delta D = D(x,y) - \bar{D}$
over an area in the image to calculate $\sigma_D$:
\begin{align}
  \sigma_D^2 = \frac{1}{4XY}\int_{-X}^{X}\int_{-Y}^{Y}\Delta D^2(x,y)dxdy.
\end{align}
Similar equations can be written in terms of PV and exposure to the detector, $E$. 

\begin{align}
  \sigma_{\text{PV}}^2 &= \frac{1}{4XY}\int_{-X}^{X}\int_{-Y}^{Y}\Delta \text{PV}^2(x,y)dxdy\\
    \sigma_{\text{E}}^2 &= \frac{1}{4XY}\int_{-X}^{X}\int_{-Y}^{Y}\Delta \text{E}^2(x,y)dxdy
\end{align}
Then by Eqs [10] and [11]''
\begin{align}
  \Delta D = CG\log_{10}e = G(\log_{10}e)\frac{\Delta E}{\bar{E}},
\end{align}
but
\begin{align}
  E &= kN \ \text{and}\\
  \sigma_E^2 &= k^2\sigma_N^2
\end{align}
where $N$ is the number of photons, which is $N = A\phi$, where $A$ is the
cross-sectional area and $\phi$ is the fluence.

Eq. [14] becomes:
\begin{align}
  \sigma_D^2 = G^2(\log_{10}^2 e ) \frac{k^2 \sigma_N^2}{k^2\bar{N}^2} = G^2(\log^2_{10}e)\frac{\sigma_N^2}{\bar{N}^2}
\end{align}

For Poisson statistics,
\begin{align}
  \sigma_N^2 = \bar{N} = A\bar{\phi} \text{and}
  \frac{\sigma^2_N}{\bar{N}^2} = \frac{1}{\bar{N}} = \frac{1}{A\bar{\phi}}. 
\end{align}

Inserting Eq. [19] into [17] gives:
\begin{align}
  \sigma_D^2 = \frac{G^2(\log^2_{10} e)}{A\bar{\phi}}.
\end{align}

For a digital system, $\text{PV} = GE$ and therefore
\begin{align}
  \Delta\text{PV} = G\Delta E. 
\end{align}
Now using Eqs. [13] and [20]
\begin{align}
  \sigma_{PV}^2 = G^2\sigma_E^2.
\end{align}
Using Eqs [16], [17], [18] and [21]
\begin{align}
  \sigma_{\text{PV}}^2 = G^2k^2\sigma_N^2 = G^2k^2A\bar{\phi}
\end{align}

Note $\sigma_D^2 \propto \frac{1}{\bar{\phi}}$, but $\sigma_{\text{PV}}^2 \propto \bar{\phi}$.

Note in real imaging systems, there are other noise sources. In particular, in
a screen-film system there is noise due the finite size and number of the silver
grains in the developed film; and in a digital system there is electronic noise
from the device that captures the light (indirect detectors) or the electrons
(direct detectors). These are usually small compared to quantum noise, at low
spatial frequencies. More about this in the lecture on noise.

SNR (ignoring image blurring and considering only x-ray quantum noise)

For screen-film systesms, using Eqs. [10] and [19]
\begin{align}
  \text{SNR}_{\text{film}} = \frac{\Delta D}{\sigma_D} = \frac{GC(\log_{10}e)}{\sqrt{\frac{G^2(\log_{10}^2e)}{A\bar{\phi}}}} = C\sqrt{A\bar{\phi}},
\end{align}
which is the same as Eq. [3]. 

For a digital system, using Eqs. [15], [20] and [22]
\begin{align}
  \text{SNR}_{\text{digital}} = \frac{\Delta \text{PV}}{\sigma_{PV}} = \frac{G\Delta E}{\sqrt{G^2 k^2 A\phi}} = \frac{k\Delta N}{\sqrt{k^2 A\phi}} = \frac{A\Delta \phi}{\sqrt{A\phi}} = C\sqrt{A\bar{\phi}},
\end{align}
which is again Eq. [3].

Non-ideal Detectors

Assume the imaging system is linear or linearizable. Further assume $w_{in}(u)$
is the input stimulus where $u$ is an independent variable and $w_{out}(u)$ is
the output response of the system. If there are two inputs, which produce two
outputs, that is: $w'_{out}(u) = w'_{in}(u)$ and $w''_{out}(u) = w''in(u)$, the
system is said to be linear if, when both inputs are applied together, $w_{in}(u) =
w'in(u) + w''in(u)$, the output is given by: $wout(u) = w'out(u) + w''out(u)$.

For a real (non-ideal) imaging system, the input maybe localized to a location
$u_0$, the response at the output is spread over a range of $u$ centered on
$u0$. Conversely, any point at the output will depend on input stimuli over a
range of positions at the input, that is:
\begin{align*}
  w_{out}(u) = \int_{-\infty}^{\infty} p(u, u') w_{\text{in}}(u')du'
\end{align*}
Let $w_{\text{in}}(u) = \delta(u - u_0)$. 


Therefore,


\subsection{Characterizing An X-ray Imaging System}
\subsubsection{Contrast}
\subsubsection{Resolution}
\subsubsection{Noise}
\subsection{Effect of Scatter}


\subsubsection{On Radiation Contrast}
\fig{img/scatter.png}{1.0}{Left: Radiograph of the skull with contrast. Right:
  Radiograph of the same skull with an anti-scatter grid in place. Note the
  improvement in image contrast.}{scatter}
% I'm assuming an anti-scatter grid is the difference between left and right
% images, I could be wrong...

The derivation of expressions for contrast and SNR under the Rose model in
section 2.2 neglected the effects of scattered radiation. In general,
scattered radiation always works to decrease the contrast and overall
quality of an image, as seen in Figure \ref{fig:scatter}.
This effect can be quantitatively demonstrated with a simple extension
of the Rose model. 

% Will make a different figure in the future similar to the one in section 2.2
\fig{img/rose_with_scatter.png}{0.3}{Rose model with scatter.}{rose_with_scatter}

Recall our previous definition of contrast under the Rose model,
\begin{align}
  C = \frac{\Delta \phi}{\bar{\phi}} = \frac{\phi_1 - \phi_2}{(\phi_1 + \phi_2)/2}
\end{align}
Referring to Figure \ref{fig:rose_with_scatter}, we can define a
``no scatter'' contrast, $C_{NS}$, accounting only for contrast from
the primary fluence components, $\phi_{1p}$ and $\phi_{2p}$, That is,
\begin{align}
  C_{NS} = \frac{\phi_{1p} - \phi_{2p}}{(\phi_{1p} + \phi_{2p})/2} = \frac{\phi_{1p} - \phi_{2p}}{P}
\end{align}
where we define $P$ as the mean primary fluence, $(\phi_{1p} + \phi_{2p})/2$.

Likewise, we can define a ``scatter'' contrast, $C_S$,
\begin{align}
  C_S &= \frac{(\phi_{1p} + \phi_{1s}) - (\phi_{2p} + \phi_{2s})}{[(\phi_{1p} + \phi_{1s}) + (\phi_{2p} + \phi_{2s})]/2}\\
      &= \frac{(\phi_{1p} - \phi_{2p}) + (\phi_{1s} - \phi_{2s})}{[(\phi_{1p} + \phi_{2p}) + (\phi_{1s} + \phi_{2s})]/2}\\
\end{align}

If we assume that the scatter component of each fluence is roughly equal, that is $\phi_{1s} \approx \phi_{2s} = \phi_S$, and define $S = \phi_S$, then $C_S$ can be written
\begin{align}
  C_S = \frac{\phi_{1p} - \phi_{2p}}{P + S}
\end{align}
and we see that scatter reduces contrast.

We can relate the scatter and no-scatter contrasts as follows:
\begin{align}
  C_S &= \frac{\phi_{1p} - \phi_{2p}}{P + S}\left(\frac{P}{P}\right)\\
      &= C_{NS}\left(\frac{P}{P + S}\right)\\
      &= C_{NS}\left(1 - \frac{S}{P + S}\right) \\
      &= C_{NS}(1 - \text{SF}) = C_{NS}\left(\frac{1}{1 + S/P}\right)
\end{align}
using the scatter fraction:
\begin{align}
  \text{SF} = \frac{S}{P + S}
\end{align}
and scatter-to-primary ratio, $S/P$. 


\subsection{Effect of Focal Spot Size}


\pagebreak
\end{document}