\documentclass[mphy386-notes.tex]{subfiles}
\begin{document}
\section{X-ray Imaging}
\subsection{X-ray Imaging Physics}
\subsubsection{Sources}
X-ray tubes.
kVp, mA, etc.
Focal spot size

\subsubsection{Interactions}
Review of interactions in x-ray energy regime
Scatter, grids
\subsubsection{Detectors}
The role of the x-ray detector is to convert the x-ray image into a visible
image, and it should do so without altering it. Ideally, the x-ray detector
should absorb all incident x-ray quanta.

The fraction of incident x-ray photons that interact in the detector, $\eta$, is
called the quantum detection efficiency and is given by
\begin{align}
  \eta = 1 - e^{-\mu t},
\end{align}
where $\mu$ is the linear attenuation coefficient of the x-ray detector and $t$ is
its thickness.

We desire to have $\mu t$ to be as large as possible. Therefore, the x-ray
detector should be thick and made of a material with high atomic number and high
physical density.

SNR for a detector taking into account the quantum detection efficiency is
\begin{align}
  \text{SNR} = C\sqrt{\eta A\bar{\phi}}.
\end{align}

\fig{img/sem.png}{1.0}{Test}{sem}

\emph{Types of X-ray Detectors}

The two major classes of x-ray detectors are analog and digital.

Analog images are acquired and displayed on film. Film has a non-linear response
to x-ray exposure. As we shall see, this presents many interesting
phenomena. There are two types of analog detectors. Film Roentgen when making
the first x-ray image used film. Film is made of silver bromide, and only
approximately 3\% of incident x-rays interact in the film. See additional
material on how film works.

Screen film systems

To reduce the x-ray dose to the patient, films were used with florescence
screens, which contain an x-ray phosphor. The x-ray phosphor converts the
x-rays into visible light (plus infrared and ultraviolet), which exposes the
film (see figure below). When the film is developed, the x-ray image appears.
The film is used to both record and display the image. As we shall see, image
quality can be compromised for screen-film systems compared to digital systems.


The output of the film image is measured in film optical density (OD).  It is a
measure of light transmission through the film.
\begin{align}
  \text{OD} = -\ln\left(\frac{I}{I_0}\right).
\end{align}

Higher OD means lower transmission and a darker film, which corresponds to a
higher x-ray exposure (see characteristic curve to the left). The characteristic
is very dependent on the film processing conditions, which can be quite
variable. The exact proportion of chemicals and the exact temperature of the
developer will affect the shape of the characteristic curve and the speed point.
Care is taken to keep the chemistry and the temperature of the developer as
constant as possible.

\fig{img/hd-curve.png}{1.0}{Test}{hd}

An important property of screen-film systems is known as reciprocity law
failure.  If a certain exposure is required to reach a certain film OD, then if
the same exposure is given but over a longer period of time, the film will have
a lower OD.  This is a basic property of film that is exposed using light.  Film
requires 3-10 light quanta to produce approximately 3-10 silver atoms in order
for the grain to be developed grain (“How Film Works” handout).  If the flux of
light is too low, then after receiving fewer than the threshold number of light
quanta, the silver atoms, being unstable can ionize before additional silver
atoms are produced.

DIGITAL

Images are acquired in digital form and are usually displayed electronically,
although they can be printed on film. Thus, the acquisition of the image is
separate from the display of the image and each component can be optimized
separately. Digital systems are usually respond linearly to x-ray exposure.
The output of a digital image is in pixel value.

3. Indirect digital detectors Phosphor (often CsI) converts x rays to light and
it is coupled to a CCD camera or a thin-film transistor (TFT) wafer. The latent
image is then digitized and transferred to a computer for processing and display

4. Direct digital detectors These use materials (e.g., Se, a-Si, CdZnTe) that
produce electron-hole pairs that can be collected directly. The an electric
field is applied across the width of the detector and the electron-hole pairs
follow the field lines that are perpendicular to the surface of the detector.

5. Photon counting detectors. It is possible for certain types of direct
digital detectors to act as photon counting systems, where each individual x-ray
interaction is count directly. Non-photon counting detectors integrate quanta
(photons or electrons) over the total exposure time of the image acquisition.

6. Computed radiography (CR) [as oppose to digital radiography (DR)] systems
These store a latent image as electrons trapped in the bulk of the phosphor
(usually BaFCl). The electrons are subsequently read out by scanning the
phosphor with a laser beam. The laser light stimulates the trapped electrons
back into the conduction band where they can return back to the valence band
with the emission of light. This light is collect, usually using a PMT.
Wavelength of the laser readout is lower than the light that is emitted by the
phosphor.

5. Image Quality Properties of X-ray Detectors

a. Contrast

The contrast in the image incident on the detector is given (ignoring any
scattered radiation) is given by the radiation contrast, Eq. [4]. We want to
know the contrast in the image called the radiographic contrast. To determine
this, we need to know how x-ray exposure incident on the detector is converted
to a visible image. This relationship between the input exposure and the output
image is given by the characteristic curve.

b. Characteristic Curve

Gives the relationship between the detector output and the exposure to the
detector. For a digital detector the characteristic curve is linear. That is,
, where PV is the pixel value in the image and G is the slope of the
characteristic curve. Further:
\begin{align}
  \frac{\Delta E}{\bar{E}} = \frac{\Delta \text{PV}}{\bar{\text{PV}}} = C.
\end{align}

Therefore, for a digital detector, the radiographic contrast is equal to the
radiation contrast and this is true for all exposure values. It is independent
of the slope of the characteristic curve.

For screen-film systems, the characteristic curve is called the H\&D curve, named after Hurter and Driffield, and it is a plot OD versus log relative exposure to the screen.

For screen-film systems, the response is non-linear and there is a toe, a
shoulder region, and a linear region in between. Base+fog is the minimum OD due
to the transparency of the film base and any darkening of the film due to
thermal effects. The net OD is the gross OD minus the base+fog level.

Characteristic curve for a screen-film system depends on the properties of the
screen-film system and the film processing (developer) conditions.

For a screen-film system, the radiographic contrast is given by difference in
optical density, D. It depends on the radiation contrast and the slope of the
H\&D curve, called gamma (G). Radiographic contrast in a screen-film image is
given by:
\begin{align}
  G &= \frac{\Delta D}{\Delta(\ln E)}\\
  \Delta D = G(\log_{10}e)\Delta(\ln E) = G(\log_{10}e)\frac{\Delta E}{\bar{E}} = CG\log_{10}e, 
\end{align}
since $\phi = kE$, then $\frac{\Delta E}{\bar{E}} = \frac{\Delta \phi}{\bar{\phi}} = C =$ radiation contrast. 

Since the characteristic curve is not linear, the exposure to the detector is
very important. The image can be properly exposed, but also under or over
exposed, where the radiographic contrast will be low (because G is low).  For
digital system, this is not a problem (at least in terms of contrast) as
illustrated below.

Effect of characteristic curve shape.  Top is for screen-film, which have a
non-linear response. Bottom is for a digital system with a linear response.
This figure only illustrates the effect on radiographic contrast and not noise
nor SNR.

c. Speed 

Speed is defined as the reciprocal of the exposure required to reach a net OD of
1.0. The speed point is considered the exposure to give a properly exposed
image. A fast system has high speed and slow system has low speed.  Screen-film
systems have an optimum exposure that must be used in order to produce a useful
image.

Factors Affecting Speed
1.  X-ray absorption by the screen
- phosphor type (atomic number, k-edge energy)
- thickness and packing density
- x-ray energy
- crystal size
2.  Conversion Efficiency of Screen (fraction of x-ray energy converted in optical energy)
- physical properties of phosphor
- optical properties of screen
- concentration of activator atoms
- x-ray energy
3.  Film Sensitivity
- silver content
- sensitizers
- film gain size, structure, etc.
4.  Matching of light emission of the screen to the spectral sensitivity of the film
5.  Film processing

d.	Latitude

For screen-film systems, since the curve is non-linear, the system has limited
latitude. Latitude refers to the range in exposure that will produce density
within the accepted range for diagnostic radiology (usually considered to be
0.25 to 2.0). Latitude does not apply to digital systems. For screen-film
systems, there is a tradeoff in latitude and contrast. Generally speaking,
systems with high contrast (large G) have limited latitude and vice versa.

For the image on the right, System A has higher speed and wider latitude than
System B. System B has higher contrast, but limited latitude.

Wide latitude is important for imaging tasks where there are large difference in
tissue types. For example a chest image requires that image display lung tissue
(mostly air) and ribs (bone). Wide latitude is required to image both of these
simultaneous with good contrast. With a digital detector, since the response to
x-ray exposure is linear, the display of the image can be manipulated so that
bone can be displayed properly and then lung tissue; or image processing can be
used so that both are imaged optimally in a single image.

e. Resolution In a phosphor screen, x rays are converted to optical photons that
must travel through the bulk of the screen to escape. For screens that are
composed of crystals of phosphor in a binder material (turbid screen), the light
is scattered multiple times as illustrated below and light can be absorbed in
the screen. The light at the output of the screen is spread over a finite area,
reducing spatial resolution. The scattering of light in the screen increases
spatial resolution because it preferentially reduces light photons that travel a
long distance. Recall that resolution can be characterized by the point spread
function (psf) and the modulation transfer function (MTF). The image below
gives a qualitative depiction of how the scattering of light broadens the psf
and thus reduces the high frequency components of the MTF.

The spatial resolution is reduced (more spread of light) as the thickness of the
screen increases. The further the distance the light needs to travel to exit
the screen, the broader the psf will be. In many instances a film is sandwiched
between two thinner screens rather than be used with a single thick screen.
This can improve the resolution compared to using a single thick screen. It is
important that the screen and film be in close contact, as any space (poor
contact) will increase the area over which the light has spread.

In CsI phosphor, the crystals of CsI form long needle shaped structures. These
“needles” act like an optical fiber reducing the lateral spread of light
improving the resolution compared to turbid screens of equal thickness.

For direct digital detectors, the spatial resolution can be very high. An
electric field can be placed across the photoconductor forcing the electrons to
travel in direction perpendicular to the surface of the detector greatly
reducing the lateral spread of the electrons.

f. X-ray Quantum Noise

The signal in a screen-film system, the signal is radiographic contrast, as
given in Eq. [9]. The noise in as screen-film image is $\sigma_D$, and it is
related to the noise in the x-ray image incident on the detector, $\sigma_E$.

For a uniform exposure, we can average the square of $\Delta D = D(x,y) - \bar{D}$
over an area in the image to calculate $\sigma_D$:
\begin{align}
  \sigma_D^2 = \frac{1}{4XY}\int_{-X}^{X}\int_{-Y}^{Y}\Delta D^2(x,y)dxdy.
\end{align}
Similar equations can be written in terms of PV and exposure to the detector, $E$. 

\begin{align}
  \sigma_{\text{PV}}^2 &= \frac{1}{4XY}\int_{-X}^{X}\int_{-Y}^{Y}\Delta \text{PV}^2(x,y)dxdy\\
    \sigma_{\text{E}}^2 &= \frac{1}{4XY}\int_{-X}^{X}\int_{-Y}^{Y}\Delta \text{E}^2(x,y)dxdy
\end{align}
Then by Eqs [10] and [11]''
\begin{align}
  \Delta D = CG\log_{10}e = G(\log_{10}e)\frac{\Delta E}{\bar{E}},
\end{align}
but
\begin{align}
  E &= kN \ \text{and}\\
  \sigma_E^2 &= k^2\sigma_N^2
\end{align}
where $N$ is the number of photons, which is $N = A\phi$, where $A$ is the
cross-sectional area and $\phi$ is the fluence.

Eq. [14] becomes:
\begin{align}
  \sigma_D^2 = G^2(\log_{10}^2 e ) \frac{k^2 \sigma_N^2}{k^2\bar{N}^2} = G^2(\log^2_{10}e)\frac{\sigma_N^2}{\bar{N}^2}
\end{align}

For Poisson statistics,
\begin{align}
  \sigma_N^2 = \bar{N} = A\bar{\phi} \text{and}
  \frac{\sigma^2_N}{\bar{N}^2} = \frac{1}{\bar{N}} = \frac{1}{A\bar{\phi}}. 
\end{align}

Inserting Eq. [19] into [17] gives:
\begin{align}
  \sigma_D^2 = \frac{G^2(\log^2_{10} e)}{A\bar{\phi}}.
\end{align}

For a digital system, $\text{PV} = GE$ and therefore
\begin{align}
  \Delta\text{PV} = G\Delta E. 
\end{align}
Now using Eqs. [13] and [20]
\begin{align}
  \sigma_{PV}^2 = G^2\sigma_E^2.
\end{align}
Using Eqs [16], [17], [18] and [21]
\begin{align}
  \sigma_{\text{PV}}^2 = G^2k^2\sigma_N^2 = G^2k^2A\bar{\phi}
\end{align}

Note $\sigma_D^2 \propto \frac{1}{\bar{\phi}}$, but $\sigma_{\text{PV}}^2 \propto \bar{\phi}$.

Note in real imaging systems, there are other noise sources. In particular, in
a screen-film system there is noise due the finite size and number of the silver
grains in the developed film; and in a digital system there is electronic noise
from the device that captures the light (indirect detectors) or the electrons
(direct detectors). These are usually small compared to quantum noise, at low
spatial frequencies. More about this in the lecture on noise.

SNR (ignoring image blurring and considering only x-ray quantum noise)

For screen-film systesms, using Eqs. [10] and [19]
\begin{align}
  \text{SNR}_{\text{film}} = \frac{\Delta D}{\sigma_D} = \frac{GC(\log_{10}e)}{\sqrt{\frac{G^2(\log_{10}^2e)}{A\bar{\phi}}}} = C\sqrt{A\bar{\phi}},
\end{align}
which is the same as Eq. [3]. 

For a digital system, using Eqs. [15], [20] and [22]
\begin{align}
  \text{SNR}_{\text{digital}} = \frac{\Delta \text{PV}}{\sigma_{PV}} = \frac{G\Delta E}{\sqrt{G^2 k^2 A\phi}} = \frac{k\Delta N}{\sqrt{k^2 A\phi}} = \frac{A\Delta \phi}{\sqrt{A\phi}} = C\sqrt{A\bar{\phi}},
\end{align}
which is again Eq. [3].

Non-ideal Detectors

Assume the imaging system is linear or linearizable. Further assume $w_{in}(u)$
is the input stimulus where $u$ is an independent variable and $w_{out}(u)$ is
the output response of the system. If there are two inputs, which produce two
outputs, that is: $w'_{out}(u) = w'_{in}(u)$ and $w''_{out}(u) = w''in(u)$, the
system is said to be linear if, when both inputs are applied together, $w_{in}(u) =
w'in(u) + w''in(u)$, the output is given by: $wout(u) = w'out(u) + w''out(u)$.

For a real (non-ideal) imaging system, the input maybe localized to a location
$u_0$, the response at the output is spread over a range of $u$ centered on
$u0$. Conversely, any point at the output will depend on input stimuli over a
range of positions at the input, that is:
\begin{align*}
  w_{out}(u) = \int_{-\infty}^{\infty} p(u, u') w_{\text{in}}(u')du'
\end{align*}
Let $w_{\text{in}}(u) = \delta(u - u_0)$. 


Therefore,


\subsection{Characterizing An X-ray Imaging System}
\subsubsection{Contrast}
\subsubsection{Resolution}
\subsubsection{Noise}
\subsection{Effect of Scatter}
\fig{img/scatter.png}{1.0}{Test}{scatter}
\subsection{Effect of Focal Spot Size}


\pagebreak
\end{document}